\documentclass{common/cs157}
\usepackage{hyperref}
% \usepackage{clrscode}
\usepackage{tikz}
\usepackage{graphicx}
\usepackage{listings}


\usepackage{amsmath}
\usepackage{amsfonts}
\usepackage{amssymb}

\usepackage{algorithmicx}
\usepackage{algorithm}
\usepackage{algpseudocode}

\usepackage[noframe]{showframe}
\usepackage{framed}
\usepackage[shortlabels]{enumitem}

\renewenvironment{shaded}{%
  \def\FrameCommand{\fboxsep=\FrameSep \colorbox{shadecolor}}%
  \MakeFramed{\advance\hsize-\width \FrameRestore\FrameRestore}}%
 {\endMakeFramed}
\definecolor{shadecolor}{gray}{0.9}


% comment this in if you want to compile the solution key:
% \sol


\hwk{1}
\due{September 21, 2021 at 14:30}


\begin{document}

\homeworkhandin % this is in common/cs157.cls if you need to edit it

The goal of this first assignment is to test your proof writing skills. While correctness of an argument is paramount, conciseness is also \emph{very} important. In general, a simple and concise proof is \emph{better} than a long verbose one. 

\begin{problem}{1}
Provide a simple proof (no more than 3 lines for each) of whether each of these statements is true or false:
\begin{enumerate}
    \item $n^{2/3} \in o(n^2)$
    \item $10^{1000}n \in O(n\log n)$
    \item $5000n \in \omega(n)$
\end{enumerate}
\end{problem}

The following problems require you to write simple proofs using the main fundamental proof techniques (i.e., contradiction, induction, counterexample, contraposition, construction). Identify the best proof technique and write a simple proof, which should not exceed 6 lines. Longer solutions will be marked down.

\begin{problem}{2}
Argue whether the next statement is true or false: "Every positive integer is equal to the sum of two integer squares".
\end{problem}

\begin{problem}{3}
The Fibonacci numbers have a number of cute properties. Prove the following:  
\begin{enumerate}[(a)]
\item $F_{4n}$ is divisible by 3 for every $n\geq1$
\item $1 < F_{n+1}/F_{n} < 2$ for all $n > 2$
\end{enumerate}
\end{problem}

\begin{problem}{4}
Let $f : A \longrightarrow B$ and $g : B \longrightarrow C$ be functions. Prove the following statement: "If $g \circ f$ is bijective, then $f$ is injective and $g$ is surjective".
\end{problem}

\begin{problem}{5}
Prove by contraposition that If $mn$ is odd, then $m$ and $n$ are odd.
\end{problem}

\end{document}