\documentclass{common/cs157}

\hwk{1}
\due{September 20, 2022 at 14:30 ET}

\begin{document}

\homeworkhandin 

The goal of this first assignment is to test your proof writing skills. While correctness of an argument is paramount, conciseness is also \emph{very} important. In general, a simple and concise proof is \emph{better} than a long verbose one. 

\begin{problem}{1}
Provide a simple proof (no more than 3 lines for each) of whether each of these statements is true or false:
\begin{enumerate}
    \item $n^{2/3} \in o(n^2)$
    \item $10^{1000}n \in O(n\log n)$
    \item $5000n \in \omega(n)$
\end{enumerate}
\end{problem}

The following problems require you to write simple proofs using the main fundamental proof techniques (i.e., contradiction, induction, counterexample, contraposition, construction). Identify the best proof technique and write a simple proof. Excessively long solutions will be marked down.

\begin{problem}{2}
Argue whether the next statement is true or false: ``Every positive integer is equal to the sum of two integer squares".
\end{problem}

\begin{problem}{3}
Provide a simple proof for the following formula using induction:
\begin{equation*}
    1^2 + 2^2 + 3^2 + . . . + n^2 = \frac{1}{6}n(n + 1)(2n + 1)
\end{equation*}
\end{problem}

\begin{problem}{4}
Let $f : A \longrightarrow B$ and $g : B \longrightarrow C$ be functions. Prove the following statement: ``If $g \circ f$ is bijective, then $f$ is injective and $g$ is surjective".
\end{problem}

\begin{problem}{5}
Prove by contraposition that if $5n + 5$ is an odd integer, then $n$ is an even integer.
\end{problem}

\end{document}