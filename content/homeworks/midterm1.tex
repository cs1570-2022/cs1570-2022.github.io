\documentclass{common/cs157}

\mdtrm{1}
\due{October\ 25, 2022}

\begin{document}
\midtermpolicyremind{}

\begin{problem}{1 - 10 points}
Let $X$ and $Y$ be two sorted arrays in non-decreasing order, containing $n$ and $m$ elements respectively. Give an $O(\log(\max\{n,m\}))$-time algorithm to find the median of all $n+m$ elements in arrays $X$ and $Y$. 
\begin{enumerate}[(a)]
    \item Provide a succinct (but clear) description of your algorithm. You may provide pseudocode.
    \item Provide a proof of the correctness of the algorithm.
    \item Analyze the running time and memory utilization of the algorithm.
\end{enumerate}
\end{problem}

\newpage

\begin{problem}{2 - 10 points}
Provide an efficient greedy algorithm that, given a set $\{x_1,x_2,\ldots,x_n\}$ of points on the real line, determines the smallest set of  closed intervals of length three that contains all of the given points.
\begin{enumerate}
    \item[(a)] Provide a succinct (but clear) description of your algorithm. You may provide pseudocode.
    \item[(b)] Prove the correctness (optimality) of your algorithm.
    \item[(c)] Analyze the running time and memory utilization of the algorithm.
\end{enumerate}
\end{problem}
\newpage
\begin{problem}{3 - 10 points}
You are given an $n \times m$ matrix with non-negative integer values. Count the number of distinct paths to reach the last cell, $(n - 1,\ m - 1)$ of the matrix from its first cell, $(0, 0)$, such that the path has total cost $C$. The cost of a path is the sum of the entries of the cells in the path. You can only move one unit right, one unit down, or one unit right-down diagonal from any cell, i.e., from cell $(i, j)$, we can move to $(i, j+1)$ or $(i+1, j)$ or $(i+1,j+1)$. Two paths are distinct if they have at least one different step.
\begin{enumerate}
    \item[(a)] Provide a succinct (but clear) description of your algorithm. The running time of your algorithm should not exceed $O(nmC)$. You may provide pseudocode.
    \item[(b)] Prove the correctness of your algorithm.
    \item[(c)] Provide an analysis of the running time and memory utilization of the algorithm.
\end{enumerate}
\end{problem}
\newpage
\begin{problem}{4 -10 points}

 Given an $n$-bit binary integer, design a divide-and-conquer algorithm to convert it into its decimal representation. For simplicity, you may assume that $n$ is a power of 2. 

\begin{enumerate}[(a)]
    \item Provide a succinct (but clear) description of your algorithm, including pseudocode.
    \item Prove the correctness of your algorithm.
    \item Analyze the running time of your algorithm. Assume that it is possible to multiply two decimal integers numbers with at most $m$ digits in $O(m^{\log_2 3})$ time. 
\end{enumerate}

\textbf{Hint:} An $n$-bit binary integer $x$ can be expressed as $x = (x_{n - 1}, x_{n - 2}, \cdots, x_1, x_0)_2$ where $x_i \in \{0, 1\}$. Let $x_{\ell} = (x_{n/2 - 1}, x_{n/2 - 2}, \cdots, x_1, x_0)_2$ be the $(n/2)$-bit binary integer corresponding to the $(n/2)$ least significant digits of $x$. Let $x_m = (x_{n - 1}, x_{n - 2}, \cdots, x_{n/2 + 1}, x_{n/2})_2$ be the $(n/2)$-bit binary integer representing the $(n / 2)$ most significant digits of $x$. Then, $x = x_{\ell} + 2^{n/2} \cdot x_{m}$. This should suggest us a way to set up a divide and conquer strategy$\dots$ :) Careful about the number of subproblems!
\end{problem}

\end{document}